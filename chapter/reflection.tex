\chapter{Reflection}
    \label{chap:reflection}
    In this chapter, what went well and what could have been improved in the project will be discussed.  Section~\ref{sec:reflection_risk} will reflect whether any of the risks defined during the risk analysis arose, as well as any unplanned risks.  Section~\ref{sec:reflection_project} will reflect on the project as a whole.
    
    \section{Risk Reflection}
    \label{sec:reflection_risk}
        During the project, some of the risks defined in the risk analysis arose.  The project's testing phase was unable to be completed in time due to unforeseen circumstances manifesting during the implementation phase.  These circumstances included the device used for development failing and needing to be sent for repair and specific tasks in the implementation stage taking longer than expected.  To mitigate the above risk, frequent sprint reviews were scheduled to ensure these events were dealt with appropriately; in the case that tasks were taking longer than expected, it was decided to continue with these as needed, and that getting the application to a working but untested condition was more critical for the project.  As for the hardware failure, while the device was sent for repair, a replacement device was used temporarily, which slowed down development during the repair period, but it was not stopped completely.
        
        There was a risk that was unplanned during the project; this was the worldwide COVID-19 pandemic.  Fortunately, the pandemic did not affect any phases of the project. It possibly would have only affected the testing phase as physically meeting with test users would not have been possible in the traditional way.  Social distancing would have had to be enforced, and masks would have needed to be worn.  However, the time frame of the project would have remained consistent with the original plan.
        
    \section{Project Reflection}
    \label{sec:reflection_project}
        Overall, the project mostly went as planned, and the application produced is satisfactory.  Due to the testing phase remaining incomplete, the final product inevitably has many bugs, and due to this, the application's behaviour may be inconsistent. However, the features defined in the requirements analysis are all present, and the basic functionality of these features is in place.
        
        If there was an opportunity to redo the project, there are some changes that would be made.  Firstly, an extensive testing plan would be designed and implemented.  The tests would contain assertion checks for how Firebase responds to requests from the application and tests to ensure that data is passed from each view within the application correctly.  More in-depth diagrams representing the flow of data between the application and the database would be produced, particularly in the context of the database.  The application designs would also conform to iOS standards more, and the fonts and colours would be consistent with the traditional iOS design.
        
        Secondly, more code-base refactoring cycles would have taken place during the implementation phase.  More time for refactoring would make the code easier to read by others and optimise the running of the application.
        
        Finally, more time would be dedicated to the testing phase.  This would allow for user testing and enough time to create all the UI and unit tests.  Having enough time for the testing phase would drastically reduce bugs and issues within the application and bring the application a step closer to being the standard it needs to be in order to be added to the App Store.