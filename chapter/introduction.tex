\chapter{Introduction}
	\label{chap:intro}
	
    \section{Overview and Motivation}  
	\label{sec:intro_overview}
	    In recent years, there has been a massive increase in the number of applications being released to the App Store.  The vast numbers of categories now available include the productivity category, which itself contains thousands of applications.  Not only can this amount of different choices be confusing for the consumer, but it can have the opposite desired effect - people end up wasting their time setting up and inputting data into these applications rather than getting on with the work that needs to be done.  To add to that, many of these applications offer a subscription service, so the features that the user needs are locked behind a paywall.
	    
	    The purpose of this project is to mitigate these issues.  A new productivity application will be created, but it will be the only application a user needs to be productive and manage their time.  It will be completely free and straightforward to use, and it will assist the user in planning their time by using the tasks they have input and looking for free time in their calendar to suggest when to work on these tasks.
        
    \section{Aims}
	\label{sec:intro_aim}
	    In order for this project to achieve its purpose of producing an easy to use productivity and time management application, the following aims have been established:
	    \begin{enumerate}[noitemsep]
	        \item Provide a free productivity solution to both students and professionals to improve their time management skills.
	        \item Implement the suggestion feature within the application.
	        \item Implement functionality so that different items such as tasks, events, habits and reminders can be created.
	        \item Research the most suitable API's and libraries to use within the application.
	        \item Implement a quality application that is ready to be released to the App Store.
	    \end{enumerate}
	    
	    To achieve these aims, relevant topics will be covered in the following chapters.  Chapter~\ref{chap:background_research} will be covering background research and related work, and this will be useful to gain an understanding of the current competition on the market and what needs to be done in order to make the application stand out.  Section~\ref{sec:deliverables_requirements} and~\ref{sec:deliverables_design} are the requirements analysis and the design. These establish what API's and libraries are to be used and the features included within the application.
	    
	    Chapter~\ref{chap:legal_social_ethical_issues} discusses the legal, professional, social and ethical issues that the project came across, including the importance of these issues and how they were dealt with.  Chapter~\ref{chap:sdl} covers the software development model used in the project and why.  Risk analysis is also covered in this chapter.  In Chapter~\ref{chap:deliverables}, each stage of the software development life cycle is discussed, and includes the designs created for the project and an outline of the code produced in the implementation phase.  Finally, a reflection on the project will be explored in Chapter~\ref{chap:reflection}.
	    